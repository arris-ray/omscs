\documentclass[12pt, letterpaper]{article}
\usepackage{amsmath}
\usepackage[a4paper, total={7in, 10in}]{geometry}
\usepackage[ruled]{algorithm2e}
\usepackage{caption}
\usepackage[utf8]{inputenc}

\begin{document}

\section{Exercise 6.3}

\subsection{Problem}

Yuckdonald's is considering opening a series of restaurants along Quaint Valley Highway (QVH). The \(n\) possible locations are along a straight line, and the distances of these locations from the start of QVH are, in miles and in increasing order, \(m_1, m_2, \dots, m_n\). The constraints are as follows:

\begin{itemize}
    \item At each location, Yuckdonald's may open at most one restaurant. The expected profit from opening a restaurant at location \(i\) is \(p_i\), where \(p_i > 0\) and \(i = 1,2,\dots,n\).
    \item Any two restaurants should be at least \(k\) miles apart, where \(k\) is a positive integer.
\end{itemize}
    
Give an efficient algorithm to compute the maximum expected total profit subject to the given constraints.

\subsection{Solution}

% Test Case
% M =   1,  2,  3,   5,  8,   9, 10
% P =  10,  5, 20,  50, 30,  35,  2
% R = *10, 10, 20, *60, 90, *95, 95

\begin{algorithm}
    \SetKwInput{KwSubproblem}{Subproblem}
    \SetKwInput{KwRecurrence}{Recurrence}
    \SetKwInput{KwTime}{Time}
    \caption{QVH1 (\(\mathcal{M, P}, k\)), maximizes the profit of selecting locations for Yuckdonald's franchises along QVH.}
    \KwIn{
        \(\mathcal{M} = \{{m_1}, {m_2},\dots,{m_n}\}\), an ordered list of locations in miles from the start of QVH.\newline
        \(\mathcal{P} = \{{p_1}, {p_2},\dots,{p_n}\}\), a list of expected profits for locations \(m_{1 \to n}\).\newline
        \(k\) is the minimum distance in miles between Yuckdonald's locations.
    }
    \KwOut{The maximum expected profit of franchise locations along QVH.}
    \KwSubproblem{Let \(P(i)\) the maximum profit of all franchise locations up to and including \(m_i\). Let \(P(0) = 0\).}
    \KwRecurrence{\(P(i) = p_i + P(j : max(i) < i-k)\)} 
    \KwTime{\(O(n)\) due to single for loop.} 
    \(P(0) = 0\) \\
    \For{\(i = 1 \to n\)}
    {
        \(P(i) = p_i + P(j : max(i) <= i-k)\)
    }
    \Return{\(P(n)\)}
\end{algorithm}

\end{document}