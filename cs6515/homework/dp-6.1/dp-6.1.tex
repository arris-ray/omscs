\documentclass[12pt, letterpaper]{article}
\usepackage{amsmath}
\usepackage[a4paper, total={6in, 10in}]{geometry}
\usepackage[ruled]{algorithm2e}
\usepackage{caption}
\usepackage[utf8]{inputenc}

\begin{document}

\section{Exercise 6.1}

\subsection{Problem}

A \textit{contiguous subsequence} of a list \textit{S} is a subsequence made up of consecutive elements of \textit{S}. For instance, if \textit{S} is 
\begin{center}
\(5, 15, -30, 10, -5, 40, 10\)
\end{center}
then \(15, -30, 10\) is a contiguous subsequence but \(5, 15, 40\) is not. Give a linear-time algorithm for the following task:
\begin{center}
\begin{itemize}
    \item[] \textit{Input}: A list of numbers, \({a_1}, {a_2},\dots,{a_n}\).
    \item[] \textit{Output}: The contiguous subsequence of maximum sum (a subsequence of length zero has sum zero).
\end{itemize}
\end{center}
For the preceding example, the answer would be \(10, -5, 40, 10\), with a sum of \(55\).
\begin{center}
\begin{itemize}
    \item[] \textit{Hint}: For each \(j\in\{1,2,\dots,n\}\), consider contiguous subsequences ending exactly at position j.
\end{itemize}
\end{center}

\subsection{Solution}

\begin{algorithm}
    \SetKwInput{KwSubDef}{Subproblem Definition}
    \SetKwInput{KwSubRec}{Subproblem Recurrence}
    \captionsetup{labelsep=newline}
    \caption{SMS1 (\(\mathcal{S}\)), finds the contiguous subsequence \(\mathcal{L}\) with maximum sum in a list.}
    \KwIn{\(\mathcal{S}=\) A list of numbers, \({a_1}, {a_2},\dots,{a_n}\).}
    \KwOut{The contiguous subsequence, \(\mathcal{L}\), with maximum sum.}
    \KwSubDef{\(\textbf{s}(i)\) is the maximum sum of a contiguous subsequence ending at index \(i\).}
    \KwSubRec{\(\textbf{s}(i) =
        \begin{cases}
            a_i,                   & \text{if } > \textbf{s}(i-1) + a_i \\
            \textbf{s}(i-1) + a_i, & \text{otherwise }
        \end{cases}
        \)}

    \(\textbf{l}[0] = 0\)

    \(\textbf{s}[0] = 0\)

    \For {\(x_i \in \mathcal{S}\)}
    {
        \eIf{\(x_i > \textbf{s}[i-1] + x_i\)}
        {
            \(\textbf{l}[i] = i\)

            \(\textbf{s}[i] = x_i\)
        } {
            \(\textbf{l}[i] = \textbf{l}[i - 1]\)

            \(\textbf{s}[i] = \textbf{s}[i-1] + x_i\)
        }
    }

    \(i = max_i(\textbf{s})\)

    \(j = \textbf{l}[i]\)

    \(\mathcal{L} = \mathcal{S}[i:j]\)

    \Return{\(\mathcal{L}\)}
\end{algorithm}

\end{document}