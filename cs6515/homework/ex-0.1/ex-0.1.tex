\documentclass[12pt, letterpaper]{article}
\usepackage{amsmath}
\usepackage[a4paper, total={6in, 10in}]{geometry}
\usepackage[ruled]{algorithm2e}
\usepackage{caption}
\usepackage[utf8]{inputenc}

\begin{document}

\section{Fibonacci Sequence}

\begin{algorithm}
    \SetKwInput{KwSubproblem}{Subproblem}
    \SetKwInput{KwRecurrence}{Recurrence}
    \SetKwInput{KwTime}{Time}
    \caption{\(FIB1(n)\), finds the \(n^{th}\) number in the Fibonacci sequence.}
    \KwIn{An integer \(n\) indicating which number in the Fibonacci sequence to return.}
    \KwOut{The \(n^{th}\) Fibonacci number.}
    \KwSubproblem{\(F(i)\) is the \(i^{th}\) Fibonacci number.}
    \KwRecurrence{\(F(i) =
        \begin{cases}
            0,& \text{if } i = 0 \\
            1,& \text{if } i = 1 \\
            F(i-1) + F(i-2),& \text{otherwise}
        \end{cases}
        \)} 
    \KwTime{\(O(n)\)} 
    \(F(0) = 0\) \\
    \(F(1) = 1\) \\
    \For{\(i = 2 \to n\)} {
        \(F(i) = F(i-1) + F(i-2)\)
    }
    \Return{\(F(n)\)}
\end{algorithm}

\section{Longest Increasing Subsequence}

% Example
% A = 5, 7, 4, -3, 9, 1, 10, 4, 5, 8, 9, 3
% L = 1, 2, 1,  1, 3, 2,  4, 3, 4, 5, 6, 3

\begin{algorithm}
    \SetKwInput{KwSubproblem}{Subproblem}
    \SetKwInput{KwRecurrence}{Recurrence}
    \SetKwInput{KwTime}{Time}
    \caption{\(LIS1(\mathcal{A})\) finds the longest increasing subsequence (LIS) in a series of numbers, \(\mathcal{A} = \{a_1,a_2,\dots,a_n\}\).}
    \KwIn{An array, \(\mathcal{A}\), of unordered integers.}
    \KwOut{The length of the LIS in \(\mathcal{A}\).}
    \KwSubproblem{\(L(i)\) is the LIS of numbers in \(\{a_1,\dots,a_i\}\) which includes \(a_i\).}
    \KwRecurrence{\(L(i) = 1 + max_{1 \le j < i-1}(L(j) : a_j < a_i \text{ and } j < i)\)}
    \KwTime{\(O(n^2)\) due to the nested for loop.} 
    \For{\(i = 1 \to n\)} {
        \(L(i) = 1\) \\
        \For{\(j = 1 \to (i-1)\)} {
            \If{\(a_j < a_i\) and \(L(i) < (1 + L(j))\)} {
                \(L(i) = 1 + L(j)\)
            }
        }
    }
    \Return{\(max(L)\)}
\end{algorithm}

\pagebreak

\section{Longest Common Subsequence}

% Example
% X = "BCDBCDA"
% Y = "ABECBAB"
% L = 4 ("BCBA")

\begin{algorithm}
    \SetKwInput{KwSubproblem}{Subproblem}
    \SetKwInput{KwRecurrence}{Recurrence}
    \SetKwInput{KwTime}{Time}
    \caption{\(LCS1(\mathcal{X, Y})\) finds the length of the longest string which is a subsequence of \(\mathcal{X} \text{ and } \mathcal{Y}\).}
    \KwIn{Two strings, \(\mathcal{X \text{ and } Y}\).}
    \KwOut{The length of the LCS between \(\mathcal{X \text{ and } Y}\).}
    \KwSubproblem{For \(i \text{ and } j\) where \(0 \le i < n\) and \(0 \le j < n\), let \(L(i,j) = \text{length of LCS in } \{x_1,\dots,x_i\} \text{ and } \{y_1,\dots,y_i\}.\)}
    \KwRecurrence{
        \(L(i,j)_{i \ge 1, j \ge 1} = 
            \begin{cases}
                max(L(i,j-1), L(i-1,j)), & \text{if } x_i \neq y_j \\ 
                1 + L(i-1,j-1), & \text{otherwise}
            \end{cases}
        \)}
    \KwTime{\(O(n^2)\) for the nested for loop.} 
    \For{\(i = 1 \to n\)} {
        \(L_{i,0} = 0\)
    }
    \For{\(j = 1 \to n\)} {
        \(L_{0,j} = 0\)
    }
    \For{\(i = 1 \to n\)}
    {
        \For{\(j = 1 \to n\)}
        {
            \eIf{\(x_i = y_j\)} {
                \(L_{i,j} = 1 + L(i-1, j-1)\)
            } {
                \(L_{i,j} = max(L(i, j-1), L(i-1, j))\)
            }
        }
    }
    \Return{\(L(n,n)\)}
\end{algorithm}

\end{document}